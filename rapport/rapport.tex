\documentclass[11pt]{article}






\usepackage[utf8]{inputenc} % encode en UTF8

\usepackage[T1]{fontenc}





\usepackage[top=2cm, bottom=2cm, left=3cm, right=3cm]{geometry}

\usepackage{graphicx}
\usepackage{caption}
\usepackage{subcaption}
\usepackage{float}
\graphicspath{ {./photo/} }



\usepackage{tikz}
\usepackage{circuitikz}




\newcommand{\HRule}{\rule{\linewidth}{0.5mm}} % Epaisseur des lignes horizontales

\pagestyle{empty}
\newcommand{\myTitle[1]}{
\begin{minipage}{.47\textwidth}
\centering
\begin{flushleft}
\includegraphics[width=0.75\textwidth]{./photo/logo_univ}
\end{flushleft}
\end{minipage}
\begin{minipage}{.47\textwidth}
\centering
\begin{flushright}
\hspace*{1cm} \includegraphics[width=0.85\textwidth]{./photo/logo_polytech_nantes}
\end{flushright}
\end{minipage}~\\[1.5cm]

\begin{center}
\vspace*{\stretch{0.3}}
\end{center}

\begin{center}
  \Large
  \vspace*{\stretch{1}}
  \HRule \\[0.2cm]
  \begin{center}
    \huge
    \textbf{#1}\\ %Titre
  \end{center}
  \textbf{\\ Projet Transversal}\\ %matière
  \HRule \\[1.5cm]
  
%  \vspace*{\stretch{5}}
%  
%  \small
%  \noindent Rédigé par \\
%  \vspace*{\stretch{0.2}}
%  \large
%  \noindent \Redac\\
%  \vspace*{\stretch{0.5}}
 
\end{center}

\vspace*{\stretch{2}}

\begin{center}\large
	\textsc{École Polytech de Nantes}\\
	\textsc{Département Électronique et Technologie du Numérique}
\end{center}

\vspace*{\stretch{5}}

\begin{center}
	\begin{minipage}{0.4\textwidth}
		\begin{flushleft} \large
        	\noindent Rédigé par\\ %\Redac
			Victor DUFRENE \\
			Anthelme BOUTRY\\
			Louison GOUY\\
			Kedai WANG\\
			Huanqing LIN
		\end{flushleft}
	\end{minipage}
	\begin{minipage}{0.4\textwidth}
		\begin{flushright}\large
			Client : Loïc MARTIN\\
			Professeur encadrant : \\Yann MAHE\\ Tchanguiz RAZBAN\\
						
		\end{flushright}
	\end{minipage}
\end{center}
}

\begin{document}
\myTitle[Rapport de mini-projet \\ - Électronique Hautes Fréquences - ]
\newpage

\section{Résumé}


\newpage

\section{Synthèse de filtre passe-bas en technologie }


L'objectif est de réaliser un filtre passe-bas dont le gabarit est donné à la figure \ref{fig:gabarit_PB} avec une réponse de Tchebychev.

\begin{figure}[H]
	\centering
	\includegraphics[width=15cm]{photo/gabarit_PB.png}
	\caption{Architecture de la balise}
	\label{fig:gabarit_PB}
\end{figure}


La conception d'un filtre hautes fréquences débute par une synthèse classique avec des éléments passifs (condensateurs et bobines) comme il a été fait dans le module moyennes fréquences du S7. A partir du gabarit de la figure \ref{fig:gabarit_PB}, l'équation ci-dessous nous est permet d'obtenir l'ordre du filtre à concevoir.



\begin{equation}
	n \geq \frac{argch(\sqrt{\frac{\alpha_{min}-1}{\alpha_{max} -1}})}{argch(1/k)}
\end{equation}


Où :
\begin{itemize}
	\item n est l'ordre du filtre souhaité;
	\item k est la sélectivité égal à$\frac{fc}{1,5fc} = \frac{2}{3} \approx 0,67$;
	\item $\alpha_{min}$ est l'atténuation minimale (en linéaire) du signal dans la bande atténué égal à $10^{\frac{15}{10}} \approx 31,6$
	\item $\alpha_{max}$ est l'atténuation maximale (en linéaire) du signal dans la bande passante égal à $10^{\frac{0,1}{10}} \approx 1,02$
\end{itemize}

L'application nous donne $n \geq 4,5$ soit au minimum un ordre 5 pour respecter le gabarit souhaité. Le schéma passe-bas en impédance que nous allons utiliser est donné ci-dessous par la figure \ref{fig:gabarit_PB}.

\begin{figure}[H]
	\centering
	\begin{circuitikz}
		\draw (0,0)
		to[V,v=$U_e$] (0,2) % The voltage source
		to[L=$g_1$] (2,2)
		to[C=$g_2$] (2,0) % The resistor
		(2,2)to[L=$g_3$] (4,2)
		to[C=$g_2$] (4,0) % The resistor
		(4,2)to[L=$g_3$] (6,2)
		to [R=$r$] (6,0) 
		to[short] (0,0);
	\end{circuitikz}
	\caption{Schéma prototype d'un filtre passe-bas de Tchebycheff en impédance.}
\end{figure}

Avec :
\begin{itemize}
	
	\item les coefficients $g_k$, les valeurs normalisées des condensateurs et des bobines;
	\item r, la résistance de charge normalisée et égale à 1. Sa valeur dénormalisé est de $50\Omega$;
	\item $U_e$, le signal d'entrée ayant une résistance série normalisée égale à 1 et donc $50\Omega$ en dénormalisée.
\end{itemize}

Pour obtenir les valeurs de $g_k$ nous ne pouvons utilisé les tableaux données dans le polycopié d'électronique moyennes fréquences car ils ne sont valables uniquement pour des ondulations de 0.5 dB et 1 dB. Notre gabarit nous impose une ondulation maximale de 0.1 dB dans la bande passante. Pour obtenir des coefficient $g_k$ qui permettent de respecter l'ondulation voulue nous avons réalisé un script sur Octave disponible en annexe. Le résultat obtenu est le suivant.

\begin{table}[H]
	\centering
	\begin{tabular}{|p{2cm}|p{2cm}|p{2cm}|p{2cm}|p{2cm}|}
		\hline
		$g_1$ & $g_2$ & $g_3$ & $g_4$ & $g_5$\\
		\hline
		1.1468 & 1.3712 & 1.9750 & 1.3712 & 1.1468\\
		\hline
	\end{tabular}
	\caption{Valeurs des composants normalisées pour une ondulation de 0.1 dB}
	\label{tab:coefficient_g_k_passe_bas}
\end{table}
 
Pour trouver les valeurs réelles des composants il faut : pour les condensateurs, multiplier les $g_k$ pairs par un coefficient $C_{denom}$ et pour les bobines, multiplier les $g_k$ impairs par un coefficient $L_{denom}$. Les valeurs de ces coefficients sont données ci-dessous.

\begin{equation}
	C_{denom} = \frac{1}{2\pi f_c R}
	\qquad
	L_{denom} = \frac{R}{2\pi f_c}
\end{equation}
Une application numérique où $f_c = 2GHz$ et $R = 50\Omega$ donne :

\begin{equation}
	C_{denom} = 1.59 pF
	\qquad
	L_{denom} = 3.98 nH
\end{equation}

Une fois le coefficient de dénormalisation appliqué nous obtenons les valeurs suivantes.

\begin{table}[H]
	\centering
	\begin{tabular}{|p{2cm}|p{2cm}|p{2cm}|p{2cm}|p{2cm}|}
		\hline
		$L_1$ & $C_2$ & $L_3$ & $C_4$ & $L_5$\\
		\hline
		4.56 nH & 2.18 pF & 7.86 nH & 2.18 pF & 4.56 nH\\
		\hline
	\end{tabular}
	\caption{Valeurs réelles des composants du filtre passe-bas.}
	\label{tab:valeurs_composant_passe_bas}
\end{table}

\end{document}